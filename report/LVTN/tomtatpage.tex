\begin{titlepage}
\centering
	{\scshape\LARGE \textbf{TÓM TẮT} \par}
	\vspace{1cm}
	
\begin{flushleft}
Trực quan hoá dữ liệu được hiểu là cách dùng hình ảnh để biểu diễn thông tin, khai thác hệ thống thị giác con người để cung cấp một cách trực quan, nhanh chóng, độc lập với ngôn ngữ để trình bày và hiển thị dữ liệu. Đây là công cụ đắc lực để phân tích và tìm hiểu thông tin một cách dễ dàng. Trực quan dữ liệu ngày càng đóng vai trò quan trọng trong mọi lĩnh vực đời sống: Sử dụng trong nghiên cứu, trong những công trình liên quan tới xử lý dữ liệu, trong các giao dịch bởi người dùng phổ thông, và đặc biệt
đối với các nhà phân tích dữ liệu.\par

Với mong muốn mang trực quan dữ liệu lên nền tảng Web và cung cấp một thư viện giúp các lập trình viên có thể dễ dàng hiện thực các ứng dụng liên quan đến trực quan dữ liệu. Thông qua việc nghiên cứu các công trình và nền tảng cần thiết như D3.js và OpenLayers chúng tôi đã áp dụng và xây dựng nên thư viện C9js. Thư viện C9js giúp các lập trình viên dễ dàng trực quan các loại dữ liệu khác nhau dưới dạng biểu đồ hay bản đồ và tương tác trên các thành phần.\par

Thư viện C9js được phát triển theo mô hình tích hợp và triển khai liên tục, là một hướng tiếp cận mới trong phát triển phần mềm. Việc đánh giá thư viện C9js được chúng tôi thực hiện dựa trên các chương trình kiểm thử và so sánh với các thư viện đã có.\par

Từ kết quả hiện thực, thư viện C9js hiện tại đã được triển khai và mong muốn nhận được sự đóng góp và phản hồi từ người dùng. Đây cũng là mong muốn của chúng tôi khi thực hiện đề tài này.\par
\end{flushleft}


\vfill % Fill the rest of the page with whitespace
\end{titlepage}